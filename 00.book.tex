\documentclass[pdftex,12pt, oneside]{article}

%\usepackage[paperwidth=8.5in, paperheight=13in]{geometry} % Folio
\usepackage[paperwidth=8.27in, paperheight=11.69in]{geometry} % A4

\usepackage{makeidx}         % allows index generation
\usepackage{graphicx}        % standard LaTeX graphics tool
                             % when including figure files
\usepackage[bottom]{footmisc}% places footnotes at page bottom
\usepackage[english]{babel}
\usepackage{enumerate}
\usepackage{paralist}
\usepackage{float}
\usepackage{gensymb}  
\usepackage{listings}
\usepackage{color}
\usepackage{mathtools} % atau \usepackage{amsmath}
\renewcommand{\baselinestretch}{1.5}

\newcommand{\HRule}{\rule{\linewidth}{0.5mm}}

\definecolor{codegreen}{rgb}{0,0.6,0}
\definecolor{codegray}{rgb}{0.5,0.5,0.5}
\definecolor{codepurple}{rgb}{0.58,0,0.82}
\definecolor{backcolor}{rgb}{0.95,0.95,0.92}

\lstdefinestyle{mystyle}{
  backgroundcolor=\color{backcolor},
  commentstyle=\color{codegreen},
  keywordstyle=\color{magenta},
  stringstyle=\color{codepurple},
  basicstyle=\footnotesize,
  breakatwhitespace=false,
  breaklines=true,
  captionpos=b,
  keepspaces=true,
  numbers=left,
  numbersep=5pt,
  showspaces=false,
  showstringspaces=false,
  showtabs=false,
  tabsize=2
}

\lstset{style=mystyle}


\begin{document}
\sloppy % biar section ga melebar melewati kertas

\begin{center}
{\large \textit{SOURCE CODE} PROGRAM \textit{WEB SERVICE} PBB-P2.}
\\[1cm]
25 November 2016\\
Priyanto Tamami, S.Kom.
\end{center}

%\frontmatter%%%%%%%%%%%%%%%%%%%%%%%%%%%%%%%%%%%%%%%%%%%%%%%%%%%%%%


%%%%%%%%%%%%%%%%%%%%%%%%%%%%%%%%%%%%%%%%%%%%%%%%%%%%%%%%%%%%%%%%%%%%%%

\section{\textit{SOURCE CODE} PEMBUATAN TABEL}

\textit{Source code} untuk pembuatan tabel menggunakan bahasa SQL biasa, adapun tabel yang digunakan ada 4 (empat) buah dengan rincian sebagai berikut :


  \subsection{Tabel SPPT}
  
  Tabel ini sudah ada terlebih dahulu sebagai tempat untuk menampung seluruh ketetapan yang ada pada PBB-P2, \textit{source code} dari pembuatan tabel SPPT ini adalah sebagai berikut :
  
  \begin{lstlisting}
CREATE TABLE SPPT (
  KD_PROPINSI CHAR(2) NOT NULL,
  KD_DATI2 CHAR(2) NOT NULL,
  KD_KECAMATAN CHAR(3) NOT NULL,
  KD_KELURAHAN CHAR(3) NOT NULL,
  KD_BLOK CHAR(3) NOT NULL,
  NO_URUT CHAR(4) NOT NULL, 
  KD_JNS_OP CHAR(1) NOT NULL, 
  THN_PAJAK_SPPT CHAR(4) NOT NULL,
  SIKLUS_SPPT NUMBER(2,0) NOT NULL,
  KD_KANWIL_BANK CHAR(2) NOT NULL,
  KD_KPPBB_BANK CHAR(2) NOT NULL,
  KD_BANK_TUNGGAL CHAR(2) NOT NULL,
  KD_BANK_PERSEPSI CHAR(2) NOT NULL,
  KD_TP CHAR(2) NOT NULL,
  NM_WP_SPPT VARCHAR2(30) NOT NULL,
  JLN_WP_SPPT VARCHAR2(30) NOT NULL,
  BLOK_KAV_NO_WP_SPPT VARCHAR2(30),
  RW_WP_SPPT CHAR(2),
  RT_WP_SPPT CHAR(3),
  KELURAHAN_WP_SPPT VARCHAR2(30),
  KOTA_WP_SPPT VARCHAR2(30),
  KD_POS_WP_SPPT VARCHAR2(5),
  NPWP_SPPT VARCHAR2(15),
  NO_PERSIL_SPPT VARCHAR2(5),
  KD_KLS_TANAH CHAR(3) DEFAULT 'XXX' NOT NULL,
  THN_AWAL_KLS_TANAH CHAR(4) DEFAULT '1986' NOT NULL,
  KD_KLS_BNG CHAR(3) DEFAULT 'XXX' NOT NULL,
  THN_AWAL_KLS_BNG CHAR(4) DEFAULT '1986' NOT NULL,
  TGL_JATUH_TEMPO_SPPT DATE NOT NULL,
  LUAS_BUMI_SPPT NUMBER(12,0) DEFAULT 0 NOT NULL,
  LUAS_BNG_SPPT NUMBER(12,0) DEFAULT 0 NOT NULL,
  NJOP_BUMI_SPPT NUMBER(15,0) DEFAULT 0 NOT NULL,
  NJOP_BNG_SPPT NUMBER(15,0) DEFAULT 0 NOT NULL,
  NJOP_SPPT NUMBER(15,0) NOT NULL,
  NJOPTKP_SPPT NUMBER(8,0) NOT NULL,
  NJKP_SPPT NUMBER(5,2),
  PBB_TERHUTANG_SPPT NUMBER(15,0) NOT NULL,
  FAKTOR_PENGURANG_SPPT NUMBER(12,0),
  PBB_YG_HARUS_DIBAYAR_SPPT NUMBER(15,0) NOT NULL,
  STATUS_PEMBAYARAN_SPPT CHAR(1) DEFAULT '0' NOT NULL,
  STATUS_TAGIHAN_SPPT CHAR(1) DEFAULT '0' NOT NULL,
  STATUS_CETAK_SPPT CHAR(1) DEFAULT '0' NOT NULL,
  TGL_TERBIT_SPPT DATE NOT NULL,
  TGL_CETAK_SPPT DATE DEFAULT SYSDATE NOT NULL,
  NIP_PENCETAK_SPPT CHAR(9) NOT NULL,
  CONSTRAINT PK_E6 PRIMARY KEY (KD_PROPINSI, KD_DATI2, KD_KECAMATAN, KD_KELURAHAN, KD_BLOK, NO_URUT, KD_JNS_OP, THN_PAJAK_SPPT)
  FOREIGN KEY (NIP_PENCETAK_SPPT) REFERENCES PEGAWAI(NIP),
  FOREIGN KEY (KD_PROPINSI, KD_DATI2, KD_KECAMATAN, KD_KELURAHAN, KD_BLOK, NO_URUT,       KD_JNS_OP) REFERENCES DAT_OBJEK_PAJAK(KD_PROPINSI, KD_DATI2, KD_KECAMATAN, KD_KELURAHAN, KD_BLOK, NO_URUT, KD_JNS_OP),
  FOREIGN KEY (KD_KANWIL_BANK, KD_KPPBB_BANK, KD_BANK_TUNGGAL, KD_BANK_PERSEPSI, KD_TP) REFERENCES TEMPAT_PEMBAYARAN(KD_KANWIL, KD_KPPBB, KD_BANK_TUNGGAL, KD_BANK_PERSEPSI, KD_TP),
  FOREIGN KEY(KD_KLS_BNG, THN_AWAL_KLS_BNG) REFERENCES KELAS_BANGUNAN(KD_KLS_BNG, THN_AWAL_KLS_BNG),
  FOREIGN KEY(KD_KLS_TANAH, THN_AWAL_KLS_TANAH) REFERENCES KELAS_TANAH(KD_KLS_TANAH, THN_AWAL_KLS_TANAH)
);    
  \end{lstlisting}
  
  \subsection{Tabel PEMBAYARAN\_SPPT}
  
  Tabel ini pun sudah ada pada basis data yang digunakan sebagai tempat menyimpan atau mencatat transaksi pembayaran. \textit{Source code} dari pembuatan tabel ini adalah sebagai berikut :
  
  \begin{lstlisting}
CREATE TABLE PEMBAYARAN_SPPT (
  KD_PROPINSI CHAR(2) NOT NULL,
  KD_DATI2 CHAR(2) NOT NULL,
  KD_KECAMATAN CHAR(3) NOT NULL,
  KD_KELURAHAN CHAR(3) NOT NULL,
  KD_BLOK CHAR(3) NOT NULL,
  NO_URUT CHAR(4) NOT NULL,
  KD_JNS_OP CHAR(1) NOT NULL,
  THN_PAJAK_SPPT CHAR(4) NOT NULL,
  PEMBAYARAN_SPPT_KE NUMBER(2,0) NOT NULL,
  KD_KANWIL_BANK CHAR(2) NOT NULL,
  KD_KPPBB_BANK CHAR(2) NOT NULL,
  KD_BANK_TUNGGAL CHAR(2) NOT NULL,
  KD_BANK_PERSEPSI CHAR(2) NOT NULL,
  KD_TP CHAR(2) NOT NULL,
  DENDA_SPPT NUMBER(12,0),
  JML_SPPT_YG_DIBAYAR NUMBER(15,0) NOT NULL,
  TGL_PEMBAYARAN_SPPT DATE NOT NULL,
  TGL_REKAM_BYR_SPPT DATE DEFAULT SYSDATE NOT NULL,
  NIP_REKAM_BYR_SPPT CHAR(9) NOT NULL,
  CONSTRAINT PK_G1 PRIMARY KEY(KD_PROPINSI, KD_DATI2, KD_KECAMATAN, KD_KELURAHAN, KD_BLOK, NO_URUT, KD_JNS_OP, THN_PAJAK_SPPT, PEMBAYARAN_SPPT_KE, KD_KANWIL_BANK, KD_KPPBB_BANK, KD_BANK_TUNGGAL, KD_BANK_PERSEPSI, KD_TP),
  FOREIGN KEY (NIP_REKAM_BYR_SPPT) REFERENCES PEGAWAI(NIP),
  FOREIGN KEY (KD_PROPINSI, KD_DATI2, KD_KECAMATAN, KD_KELURAHAN, KD_BLOK, NO_URUT, KD_JNS_OP, THN_PAJAK_SPPT) REFERENCES SPPT(KD_PROPINSI, KD_DATI2, KD_KECAMATAN, KD_KELURAHAN, KD_BLOK, NO_URUT, KD_JNS_OP, THN_PAJAK_SPPT),
  FOREIGN KEY (KD_KANWIL_BANK, KD_KPPBB_BANK, KD_BANK_TUNGGAL, KD_BANK_PERSEPSI, KD_TP) REFERENCES TEMPAT_PEMBAYARAN(KD_KANWIL, KD_KPPBB, KD_BANK_TUNGGAL, KD_BANK_PERSEPSI, KD_TP)
);
  \end{lstlisting}
  
  \subsection{Tabel LOG\_TRX\_PEMBAYARAN}
  
  Tabel ini digunakan untuk menyimpan atau mencatat proses transaksi pembayaran yang berhasil dilakukan. \textit{Source code} untuk tabel LOG\_TRX\_PEMBAYARAN ini adalah sebagai berikut :
  
  \begin{lstlisting}    
CREATE TABLE LOG_TRX_PEMBAYARAN (	
  NOP VARCHAR2(18 BYTE) NOT NULL,
  THN VARCHAR2(4 BYTE) NOT NULL,
  NTPD VARCHAR2(30 BYTE) NOT NULL,
  POKOK NUMBER,
  NAMA_WP VARCHAR2(50 BYTE),
  ALAMAT_OP VARCHAR2(150 BYTE),
  MATA_ANGGARAN VARCHAR2(15 BYTE),
  MA_SANKSI VARCHAR2(20 BYTE),
  DENDA NUMBER,
  PEMBAYARAN_KE NUMBER(2,0),
  IP_CLIENT VARCHAR2(30 BYTE),
  CONSTRAINT LOG_TRX_PEMBAYARAN_PK PRIMARY KEY 
    ("NOP", "THN", "NTPD"));  
  \end{lstlisting}
  
  \subsection{Tabel LOG\_REVERSAL}
  
  Tabel ini digunakan untuk mencatat historis transaksi \textit{reversal} pembayaran yang telah selesai dilakukan. \textit{Source code} untuk tabel ini adalah sebagai berikut :
  
  \begin{lstlisting}
CREATE TABLE LOG_REVERSAL (	
  NOP VARCHAR2(20 BYTE) NOT NULL,
  THN VARCHAR2(4 BYTE) NOT NULL,
  NTPD VARCHAR2(30 BYTE) NOT NULL,
  IP_CLIENT VARCHAR2(30 BYTE),
  CONSTRAINT LOG_REVERSAL_PK PRIMARY KEY ("NOP", "THN", "NTPD"));    
  \end{lstlisting}


\section{\textit{SOURCE CODE STORE PROCEDURE}}

\textit{Store procedure} pada basis data akan terbagi menjadi 3 (tiga) bagian. Fungsi dari \textit{store procedure} ini adalah sekumpulan baris program yang menggunakan bahasa PL/SQL yang didukung oleh sistem basis data Oracle, baris demi baris akan dijalankan oleh sistem basis data, yang memudahkan pemindahan data antar beberapa tabel terjadi dengan sangat cepat. Ketiga bagian \textit{store procedure} yang digunakan pada sistem aplikasi ini adalah sebagai berikut :

\subsection{\textit{STORE PROCEDURE} SPPT\_TERHUTANG}



\subsection{\textit{STORE PROCEDURE} PROSES\_PEMBAYARAN}

\subsection{\textit{STORE PROCEDURE} REVERSAL\_PEMBAYARAN}


\section{\textit{SOURCE CODE BUILD TOOLS}}


\section{\textit{SOURCE CODE} KONFIGURASI application.properties}


\section{\textit{SOURCE CODE} KONFIGURASI log4j.properties}


\section{\textit{SOURCE CODE} AppConfig.java}


\section{\textit{SOURCE CODE} HibernateConfiguration.java}


\section{\textit{SOURCE CODE} SerialConstant.java}


\section{\textit{SOURCE CODE} StatusRespond.java}


\section{\textit{SOURCE CODE} SpptRestController.java}


\section{\textit{SOUCE CODE} StoreProceduresDao.java}


\section{\textit{SOURCE CODE} StoreProceduresDaoImpl.java}


\section{\textit{SOURCE CODE} AppInitializer.java}


\section{\textit{SOURCE CODE} PembayaranSppt.java}


\section{\textit{SOURCE CODE} ReversalPembayaran.java}


\section{\textit{SOURCE CODE} Sppt.java}


\section{\textit{SOURCE CODE} Status.java}


\section{\textit{SOURCE CODE} StatusInq.java}


\section{\textit{SOURCE CODE} StatusRev.java}


\section{\textit{SOURCE CODE} StatusTrx.java}


\section{\textit{SOURCE CODE} PembayaranServices.java}


\section{\textit{SOURCE CODE} PembayaranServicesImpl.java}


\section{\textit{SOURCE CODE} ReversalServices.java}


\section{\textit{SOURCE CODE} ReversalServicesImpl.java}


\section{\textit{SOURCE CODE} SpptServices.java}


\section{\textit{SOURCE CODE} SpptServicesImpl.java}


\end{document}